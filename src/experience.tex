\section{Experience of a Leadership Computing Facility}\label{olcf}

\subsection{Proliferation and Common Functionality} \label{commonFunc}

The problems with the increase in the number of existing workflow management
systems have been illustrated well by reports and discussions surrounding the
future of workflow management in the leadership computing facilities. The
proliferation of workflow management systems and lack of a consistent
definition of a workflow are significant barriers to the adoption of this
technology in these facilities. The High Performance Computing Facility
Operational Assessment 2015: Oak Ridge Leadership Computing Facility (OLCF)
report \cite{barker_scientific_2007} describes the problem that such
facilities face.  \begin{displayquote} These discussions concluded with the
observation that the current proliferation of workflow systems in response to
perceived domain-specific needs of scientific workflows makes it difficult to
choose a site-wide operational workflow manager, particularly for
leadership-class machines. However, there are opportunities where facilities
can centralize workflow technology offerings to reduce anticipated
fragmentation. This is especially true if a facility attempts to develop,
deploy, and operate each and every workflow solution requested by the user
community. Through these evaluations, the OLCF seeks to identify interesting
intersections that are of the most value to OLCF stakeholders.
\end{displayquote}  OLCF's strategy is notable because it makes a
very practical observation that the problem of proliferation can be solved by
consolidation of common functionality. This is typical of an operational
perspective where deployment of capability is more important than in-depth
investigation and research into how that capability functions.

\subsection{Interoperability}\label{interop}

There have been a number of community calls for interoperability. For example,
Session IV of the Twentieth Anniversary Meeting of the SOS Workshop (SOS20)
focused on workflow and workflow management system development activities of
the three participating institutions: Sandia National Laboratory, Oak Ridge
National Laboratory, and the Swiss National Supercomputing Centre
\cite{pack_sos20_2016}. Multiple presenters illustrated the challenges facing
the workflow science community and widely agreed that no single workflow
management system could satisfy all the needs of those present. Instead,
attendees proposed that the community as a whole would be served best by
seeking to enable interoperability where possible.

Workflow interoperability is not just a conceptual attribute, but one with
important practical implications. For example, DOE Leadership Computing
Facilities, as in \S\ref{commonFunc}, are affected by the lack of
interoperabilty of all types. Consider the possibility that every facility
may end up supporting different workflows systems entirely, so that workflows
at one facility can not be run at another without significant work to install
one or more additional workflow management systems! This idea is also
illustrated well in The Future of Scientific Workflows report through the
concept of the ``large-scale science campaign'' \cite{deelman_future_2015}.
Such a campaign integrates multiple workflows, not necessarily all in the same
workflow management system or at the same facility, to perform data
acquisition from experimental equipment, modeling and analysis with
supercomputers, and data analysis with either grid computing or
supercomputers.