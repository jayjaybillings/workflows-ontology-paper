\section{Discussion and The Road Ahead}\label{discussion}

This is a practice (experience) paper motivated by the widely shared
perception if not strong empirical evidence/observation  that  there is a
problem in the current practice of workflow management systems.

There is an important separation between challenges of expressing workflows
effectively (algorithm) versus a workflow system that will execute the
workflow (algorithm). This paper is about the  the practice of using workflow
systems and not expressing workflows effectively. Further, it is not a
theoretically motivated or survey paper about models of workflows or workflow
systems; plenty of such papers exist which have had limited impact on the
practice of workflow systems. 

This work describes the variety of workflows, technologies, problems, and
challenges commonly found in the workflow science space.  Self-evidently, no
single workflow management system will be able to address the next generation
of scientific challenges and practical experience dictates that a change is
necessary. We propose  that common Building Blocks are a promising and
practical solution. It proposes that a Building Blocks model solve problems of
system proliferation and interoperability by redeveloping common functionality
that exists in workflow management systems into reusable services.

An important and critical test will be to devise a validation (or negation)
test for hypothesis that a building blocks approach to workflows is in fact
more scalable, sustainable and better practice than monolithic workflow
systems. We do not harbour illusions that it will not be easy, or that it is
necessarily even possible.

It is illustrative if not instructive to understand the ecosystem of the
Apache BigData Software Stack/Cloud Model, where there are many seemingly
similar components for  data-intensive workflows. The proliferation of
components suggests there is an strong preference of functional specialization
and diversity of use, as opposed to interoperability.  Equivalentally, there
is a strong binding of components to platforms.

