%%This is a very basic article template.  %%There is just one section and two
%%subsections.  
\documentclass[format=acmsmall]{acmart}
\citestyle{acmauthoryear}

%\usepackage{cite} 
\usepackage{natbib}
\usepackage{tabularx}
\usepackage[autostyle]{csquotes} 
\usepackage{url} 
\usepackage{todonotes}
\usepackage{float}
\usepackage{array}
\usepackage{tabu}

\begin{document}

\title{Toward Common Components for Open Workflow Systems}

\author{Jay Jay Billings}
\orcid{orcid.org/0000-0001-8811-2688}
\affiliation{
\institution{Oak Ridge National Laboratory}
\streetaddress{PO Box 2008 MS 6173}
\city{Oak Ridge}
\state{TN}
\country{USA}
\postcode{37831}}
\email{billingsjj@ornl.gov\\ Twitter: @jayjaybillings} 

\author{Shantenu Jha}
\affiliation{
\institution{RADICAL, ECE - Rutgers University}
\city{Piscataway}
\state{NJ}
\country{USA}
\postcode{08854}}
\email{shantenu.jha@rutgers.edu}

\begin{abstract}

The role of scalable high-performance workflows and flexible workflow
management systems that can support multiple simulations will continue to
increase in importance. For example, with the end of Dennard scaling, there is
a need to substitute a single long running simulation with multiple repeats of
shorter simulations, or concurrent replicas. Further, many scientific problems
involve ensembles of simulations in order to solve a higher-level problem or
produce statistically meaningful results. However most supercomputing software
development and performance enhancements have focused on optimizing single-
simulation performance. On the other hand, there is a strong inconsistency in
the definition and practice of workflows and workflow management systems. This
inconsistency often centers around the difference between several different
types of workflows, including modeling and simulation, grid, uncertainty
quantification, and purely conceptual workflows. This work explores this
phenomenon by examining the different types of workflows and workflow
management systems, reviewing the perspective of a large supercomputing
facility, examining the common features and problems of workflow management
systems, and finally presenting a proposed solution based on the concept of
common building blocks. The implications of the continuing proliferation of
workflow management systems and the lack of interoperability between these
systems are discussed from a practical perspective. In doing so, we have begun
an investigation of the design and implementation of open workflow systems for
supercomputers based upon common components.
\end{abstract}

\maketitle

\underline{Notice of Copyright:} This manuscript has been authored by UT-
Battelle, LLC under Contract No. DEAC05-00OR22725 with the U.S. Department of
Energy. The United States Government retains and the publisher, by accepting
the article for publication, acknowledges that the United States Government
retains a nonexclusive, paid-up, irrevocable, world-wide license to publish or
reproduce the published form of this manuscript, or allow others to do so, for
United States Government purposes. The Department of Energy will provide
public access to these results of federally sponsored research in accordance
with the DOE Public Access Plan (http://energy.gov/downloads/doe-public-access-plan).

\section{Introduction}

Suppose for a moment that there is an interesting activity that would benefit
from automation, which is known because the activity exhibits the following
properties: 
\begin{itemize} 
\item The goal of the activity is known and desirable.  
\item The tasks to achieve the goal and complete the activity are also known 
and, furthermore, highly repetitive even in cases where decisions must be 
made to continue.  
\item The results of achieving this goal can be consumed or processed in 
standard ways.  
\end{itemize}

This example may be recognized by many as a description - but not a definition -
of a \textit{workflow}. Experts from many different backgrounds can easily think
of activities that fit this description and even systems that automate the
activity. However, each expert will probably also imagine a different workflow:
a businessperson might imagine the workflow for processing payments, a medical
professional might imagine it to be updating medical charts and records, and
scientists might imagine performing an analysis with modeling and simulation
software, analyzing a large amount of data, or quantifying uncertainty. Within
the scientific community this has led to a rather predictable situation:
Everyone has a different definition of ``workflow'' and has created their own
systems for managing and processing workflows.

This leads to some very practical consequences for \textit{scientific
workflows}. In spite of the similarities in high-level abstractions and
higher-order concepts, extremely specialized software solutions and communities
have developed to process scientific workflows. These differences hold across
scientific problems, all generally providing some level of service that was not
or perhaps is not available in a regular programming language, system library,
or problem solving workbench. These systems have accreted workflow management
capabilities over time that have effectively resulted in the creation of large,
monolithic software stacks that cannot communicate between each other, require a
very large amount of expertise to use, often put very high demands on back-end
systems either by design or through assumptions, and are often too specialized
to jump between workflow execution for data analysis and modeling and
simulation. 

Recent developments suggest that this may be neither desirable nor necessary.
Specifically, as scientific problems become more complex, functionality moves
from libraries into operating systems, and open source development continues to
rise as the dominant means of collaborating on software development activities,
the scalability, sophistication, and maintainability of large monolithic systems
raise significant questions. Software complexity, in particular, often makes it
impossible for development on large systems to scale to the required level 
because the accretion of new capability means managing larger pools of
people and a larger software development effort. One obvious alternative with
some degree of historical precedence in the field is to develop \textit{common
building blocks} that provide common services used to both define and execute
workflows. Such an approach not only makes it possible to coalesce around a
standard definition and understanding of workflows, but to separate and
distribute the work required to construct the building blocks from the effort to
define workflows and to create workflow management systems that may share the
building blocks while retaining required customizations. This article
contributes to the on-going discussion by providing the following:
\begin{itemize} \item We provide an illustration of the diverse nature of
scientific workflows, (see \S \ref{workflows}), that describes the different
areas where scientific workflows and systems have appeared in the literature,
how they have been classified in the past, and the arguments around coalescence
that are driven by calls of interoperability.  \item We describe a necessary
subset of functionality that is common across a number of scientific workflow
management systems which would, in principle, be good candidates for
consolidation and sharing, (see \S \ref{commonFunc}).\todo{"F your language! I
want C/Python/something!" What to say?} \item We develop an understanding of
these common elements as building blocks and how composing these building blocks
addresses a number of the problems not easily addressed by the monolithic design
of existing systems, see \S \ref{buildings-blocks}).\todo{One good example: How
to run Pegasus or BigPanda on Titan? Not easy, but would be easier with the
right building block.} \end{itemize}


\section{Background on Workflow Models}\label{workflows}

As mentioned previously, one of the most challenging aspects of studying
workflows is the way the vocabulary has been overloaded unintentionally.
It is somewhat clearer to understand them by considering the historial
perspective.

The use and study of workflows and the initial implementation of
workflow management systems developed in the business world with the
need to automate business processes. Lud\"{a}scher et al. ascribe the
origins of workflows and workflow management systems to ``office
automation'' trends in the 1970s, \cite{ludascher_scientific_2006}. Van Der
Aalst argues that ``workflows'' arose from the needs of businesses to not only
execute tasks, but ``to manage the flow of work through the
organization,'' and that managing workflows is the natural evolution
from the monolithic applications of the 1960s to applications that rely
on external functionality in the 1990s, \cite{van_der_aalst_application_1998}.
\footnote{One might argue that Van Der Aalst's depiction continues today with the
growth of the ``microservices'' architectural movement.} By 1995, in the
presence of many workflow tools, the Workflow Management Coalition had developed a ``standard'' definition of
workflows, \cite{hollingsworth_workflow_1993},

\begin{displayquote}
A Workflow is the automation of a business process, in whole or part, during
which documents, information or tasks are passed from one participant (a 
resource; human or machine) to another for action, according a set of 
procedural rules. 
\end{displayquote}

In the early 2000s, workflow systems started finding use in scientific
contexts where process automation was required for scientific uses
instead of traditional business uses. The focus of scientific workflows,
at the time, also shifted to focus primarily on data processing for
large ``grids'' of networked services, \cite{yu_taxonomy_2005}. Yu and Buyya
define a workflow as

\begin{displayquote}
... a collection of tasks that are processed on distributed resources in a
well-defined order to accomplish a specific goal.
\end{displayquote}

This latter definition is important because of what is missing: the human
element. For many in the grid/eScience workflows community this has become
the standard definition of a workflow and the involvement of humans
results not in a single workflow, but multiple workflows spanned by a human.
Machines or instruments are absent from the definition as well, but in
practice many modern grid workflows are launched automatically
when data ``comes off'' of instruments because they remain the primary
source of data in grid workflows, (c.f. - \cite{megino_panda:_2015}).

In addition to ``grid workflows,'' the scientific community started
exploring ``modeling and simulation workflows'' which focus not on data
flow, but on the orchestration of activities related to modeling and
simulation instead, sometimes on small local computers, but often on the
largest of the world's ``Leadership Class'' supercomputers. Unlike grid
workflows they tend to require human interaction in one way or another.
Some of these workflows are defined in the context of a particular way
of working, such as the Automation, Data, Environment, and Sharing
(ADES) model of Pizzi et al., \cite{pizzi_aiida:_2016}, the
``Design-to-Analysis'' model of Clay et al., \cite{clay_incorporating_2015}, or
the model of Billings et al.,\cite{billings_eclipse_2017}. Many scientific workflows, while exceptionally well defined and capable, remain hard-coded into dedicated environments - 
not general purpose workflow systems - developed for the sole purpose of
executing that single or at most a few related workflows, \cite{lingerfelt_beam:_2016}.

Additional types of workflows in the scientific community include
workflows that process ensembles of calculations, \cite{montoya_apex_2016},
and workflows that are used for testing software. \todo{EXPAND!}

\subsection{Taxonomies and
Classification}\label{taxonomies-and-classification}
\todo{Why are we talking about taxonomies and classification?}
There have been several efforts to classify, survey or develop taxonomies
for workflows. Yu and Buyya present an exceptional and noteworthy taxonomy for grid
workflows. Multiple other efforts provide highly useful vocabularies and analyses as
well.

Yu and Buyya developed a taxonomy for workflow management systems on
grids that sought to capture the architectural style while identifying
comparison criteria, \cite{yu_taxonomy_2005}. Their work is notable because it
largely avoids a discussion of applications and focuses purely on the
functional properties of the workflow management systems as they exist
on the grids. Yu and Buyya root their taxonomy on five core elements of
grid-based workflow management systems: workflow design, information
retrieval, workflow scheduling, fault tolerance, and data movement.
While many of the properties and taxonomic elements they describe seem
common to all systems, others would appear to be grid-specific at present, such
as ``Workflow QoS Constraints''. Their work also shows how thirteen common grid
workflow management systems, including Pegasus and Kepler, are covered by the
taxonomy. Like other authors, Yu and Buyya cite the lack of standardized
workflow syntax and language as sources of interoperability issues. Their work is extremely
detailed and a very helpful resource for understanding grid workflows.

As a counter point to the often automated grid workflows, human involvement is 
critical in cases which require adaptive
management, as shown by Han and Bussler, \cite{han_taxonomy_1998}. Their work
considers adaptive workflow management in the context of healthcare
workflows and argues that workflow technology in 2002 was incapable of
adapting sufficiently to meet the unexpected needs of medical personnel.
Along with unexpected needs (``changing environment''), they cite the
evolution of software systems (``technical advances'') as another
critical area that leads to required changes in workflow management
systems.

Han and Bussler also discuss situations that lead to ``ad-hoc derivation'' of
workflows. ``Ad-hoc derivation'' in this case means generating extra
workflow steps or details, such as converting from an abstract to a
concrete workflow as Pegasus and other grid workflow systems do.
Specifically, Han and Bussler cite dynamic refinement, user involvement,
unpredictable events and erroneous situations as systems that require
the workflow to behave in an unplanned way, and for which
workflow managements systems should be prepared. Meta-models, open-point
(more commonly known as ``extension point'') or hybrid approaches are
proposed as solutions.

It is important to note that Han and Bussler only consider business
workflows and management systems, not scientific workflows and
management systems. In this context they also share some important
wisdom that is, arguably, of equal importance to scientific workflows:

\begin{displayquote}
Workflow systems do not exist for their own purposes. They
are a constituent component of a business system. A business system is usually domain
specific.
\end{displayquote}

This is an important consideration for scientific workflows because the
``business logic'' of scientific workflows is ``domain logic'' that is
highly specific to the scientific domain under consideration. This
necessarily leads to a diverse workflow science ecosystem.

Scientific workflow management systems have flourished since their
inception, although not without significant overlap and duplication of
effort. The survey of scientific workflow management systems by Barker
and Hemert illustrates both growth and growing pains while also providing
important observations and recommendations on the topic,
\cite{barker_scientific_2007}.

Barker and Hemert also provide key insights into the history of
workflow management systems as an important part of business
automation. The authors make an important comparison between traditional
business workflow management systems and their scientific counterparts,
citing in particular that traditional business workflow tools employ the
wrong abstraction for scientists They define workflows using the
``standard'' definition from the Workflow Management Coalition, (c.f. \S \ref{workflows} above).

The discussion points that Barker and Hemert raise are important because
of their continuing importance and relevance today, particularly the
need to enable programmability through standard languages instead of
custom, proprietary languages. Sticking to
standards is also important and perhaps illustrated best by Barker's and
Hemert's statement that

\begin{displayquote}
If software development and tool support terminates on one proprietary 
framework, workflows will need to be re-implemented from scratch.
\end{displayquote}

This is an important point even for workflow tools that do not use
proprietary standards, but ``roll their own'' solutions. What can be
done to support those tools and reproduce those workflows once support
for continued development ends?

Notably, in their discussion about the Kepler workflow management
system, Barker and Hemert state that

\begin{displayquote}
Actors are re-usable independent blocks of computation, such as:
Web Services, database calls, etc.
\end{displayquote}

This is consistent with the solution proposed in this paper and discussed in \S \ref{}.\todo{fix reference!}

Montoya et al. discuss workflow needs for the Alliance for Application
Performance at Extreme Scale (APEX), \cite{nersc_apex_2016}, and describe
three main classes of workflows: Simulation Science, Uncertainty
Quantification (UQ), and High Throughput Computing (HTC),
\cite{montoya_apex_2016}.
HTC workflows start with the collection of data from experiments that is in turn
transported to large compute facilities for
processing. Many grid workflows are HTC workflows, but not all HTC
workflows are grid workflows since some HTC workflows - such as those
those presented by Montoya et al. - may be run on large resources that
are not traditionally ``grid machines.'' When Montoya et al. describe scientific workflows,
they are refering to the modeling and simulation workflows described above. Montoya et al. also provide a
detailed mapping of each workflow type to optimal hardware resources for the APEX
program.

The U.S. Department of Energy sponsored the \emph{DOE NGNS/CS Scientific
Workflows Workshop} on April 20-21st 2015. In the report, Deelman et al.
describe the requirements and research directions for scientific
workflows for the exascale environment, \cite{deelman_future_2015}\cite{deelman_future_2017}. The report (and paper) describes scientific workflows primarily by three application types:
Simulations, Instruments, and Collaborations. The findings of the workshop are
comprehensive and encouraging, with recommendations for research
priorities in Application Requirements, Hardware Systems, System
Software, Workflow Management System Design and Execution, Programming and Usability,
Provenance Capture, Validation, and Workflow Science.

The definitions of a ``workflow'' and ``workflow management systems''
are thoroughly explored and put into context for the purposes of the
workshop. The authors of the report are very careful to define workflows
not just as a collection of managed processes, which is common, but in
such a way that it is clear that reproducibility, mobility and some
degree of generality are required by both the description of the
workflow and the management system. \footnote{The report appears to provide
three separate definitions for ``workflow'' on pages 6, 9 and 10.}

Ferreira da Silva et al. attempt to characterize workflow management systems in 
\cite{ferreira_da_silva_characterization_nodate}. The authors reduce key properties of workflow
systems into four incongruent areas: (i) design, (ii) execution and monitoring, (iii)
reusability and (iv) collaboration. These properties are essential
considerations for most  software with limited specificity for workflow
management systems. Furthermore, there is general conflation between
classification and taxonomy and significant incoherence between entries in
equivalence classes. Most significantly, it fluctuates somewhat chaotically
between discussing workflows and workflow management systems without linking
workflow properties to successful design and properties of workflow systems.

\subsection{Interoperability}

Uncertainty in the very definition of workflows and workflow management systems makes it
difficult or impossible to address complicated workflows that may require interoperability
across computing facilities or distribution of ``sub-workflows'' across multiple systems.
For example, Session IV of the Twentieth Anniversary Meeting of the SOS Workshop
(SOS20) focused on workflow and workflow management
system development activities of the three participating institutions: Sandia
National Laboratory, Oak Ridge National Laboratory, and the Swiss National Supercomputing
Centre, \cite{pack_sos20_2016}. Multiple
presenters illustrated the challenges facing the workflow science community and
widely agreed that no single workflow management system could satisfy
all the needs of those present. Instead, attendees proposed that the community as a whole would
be served best by seeking to enable interoperability where possible. 

Similar considerations are present in reports and discussions surrounding the future of
workflow management in the Leadership Computing Facilities where the ``proliferation'' of
workflow management systems and the lack of a consistent definition of a workflow are 
signficant barriers to the adoption of this technology. The High Performance Computing Facility
Operational Assessment 2015: Oak Ridge Leadership Computing Facility (OLCF)
report, \cite{barker_scientific_2007}, illustrates the problem that such facilities
face,

\begin{displayquote}
These discussions concluded with the observation that the current proliferation
of workflow systems in response to perceived domain-specific needs of 
scientific workflows makes it difficult to choose a site-wide operational
workflow manager, particularly for the leadership-class machines. However,
there are opportunities where facilities can centralize workflow technology 
offerings to reduce anticipated fragmentation. This is especially true if a 
facility attempts to develop, deploy, and operate each and every workflow 
solution requested by the user community. Through these evaluations, the OLCF
seeks to identify interesting intersections that are of the most value to OLCF
stakeholders.
\end{displayquote}

The Oak Ridge Leadership Computing Facility (OLCF) is not alone in this
because other Leadership Computing Facilities face the same problem. Few
of these facilities have discussed the possibility every facility may end up
supporting different workflows systems entirely such that workflows at
one facility can not be run at another without significant work to
install one or more additional workflow management systems!

This idea is also illustrated well in The Future of Scientific Workflows report by Deelman et  
al. through the concept of the ``large-scale science campaign,'' \cite{deelman_future_2015}.
Such a campaign integrates multiple workflows, not necessarily all in the same workflow
management system, to perform data acquisition from experimental equipment, modeling and analysis with supercomputers, and data analysis with either grids computing or
supercomputers.\footnote{This is, in essence, the same problem that
PanDA/BigPanDA solves for the Large Hadron Collider. So, while it is not
a new problem, it is increasingly common.} An interesting case study is the ICEMAN project at Oak Ridge National Laboratory, (on which Mr. Billings is a co-principal investigator). ICEMAN
seeks to use a combination of instruments from the Spallation Neutron
Source, OLCF compute resources such as the Titan supercomputer, the
Laboratory's Compute Advanced Data Environment for Science (CADES), the
Eclipse Integrated Computational Environment for modeling and simulation workflows, and AiiDA (c.f. - \cite{pizzi_aiida:_2016}) to automate the analysis of inelastic and quasielastic neutron scattering data as part of a comprehensive problem solving environment for scientists.

The brief summary of different workflow models and related problems above illustrate the
diversity in the ``marketplace'' and illustrates the lack of a coherent understanding of 
workflows or search for higher level concepts. At this point it should be clear that there are 
two significant problems facing the workflow science community:
\begin{itemize}
  \item Everyone has their own workflow tools, and no two tool chains are the same.
  \item There is very little interoperability in the workflow tool space.
\end{itemize} The following sections propose that enough work has
been done experimentally and that the community would greatly benefit from seeking a
higher level of understanding and standardization through common building blocks.

\section{Experience of a Leadership Computing Facility}

\subsection{Proliferation and Common Functionality} \label{commonFunc}

The problems with the increase in the number of existing workflow management
systems have been illustrated well by reports and discussions surrounding the
future of workflow management in the Leadership Computing Facilities. The
``proliferation'' of workflow management systems and the lack of a consistent
definition of a workflow are signficant barriers to the adoption of this
technology in these facilities. The High Performance Computing Facility
Operational Assessment 2015: Oak Ridge Leadership Computing Facility (OLCF)
report, \cite{barker_scientific_2007}, describes the problem that such
facilities face,  \begin{displayquote} These discussions concluded with the
observation that the current proliferation of workflow systems in response to
perceived domain-specific needs of scientific workflows makes it difficult to
choose a site-wide operational workflow manager, particularly for the
leadership-class machines. However, there are opportunities where facilities
can centralize workflow technology offerings to reduce anticipated
fragmentation. This is especially true if a facility attempts to develop,
deploy, and operate each and every workflow solution requested by the user
community. Through these evaluations, the OLCF seeks to identify interesting
intersections that are of the most value to OLCF stakeholders.
\end{displayquote}  The strategy of the OLCF is notable because it makes a
very practical observation that the problem of proliferation can be solved by
consolidation of common functionality. This is typical of an operational
perspective where deployment of capability is more important than in-depth
investigation and research into how that capability functions.

\subsection{Interoperability}

There have been a number of community calls for interoperability. For example,
Session IV of the Twentieth Anniversary Meeting of the SOS Workshop (SOS20)
focused on workflow and workflow management system development activities of
the three participating institutions: Sandia National Laboratory, Oak Ridge
National Laboratory, and the Swiss National Supercomputing Centre,
\cite{pack_sos20_2016}. Multiple presenters illustrated the challenges facing
the workflow science community and widely agreed that no single workflow
management system could satisfy all the needs of those present. Instead,
attendees proposed that the community as a whole would be served best by
seeking to enable interoperability where possible.

Workflow interoperability is not just a conceptual attribute, but one with
important practical implications. For example, DOE Leadership Computing
Facilities, as in \S\ref{commonFunc}, are affected by the lack of
interoperabilty (of all types). Consider the possibility that every facility
may end up supporting different workflows systems entirely such that workflows
at one facility can not be run at another without significant work to install
one or more additional workflow management systems! This idea is also
illustrated well in The Future of Scientific Workflows report through the
concept of the ``large-scale science campaign,'' \cite{deelman_future_2015}.
Such a campaign integrates multiple workflows, not necessarily all in the same
workflow management system or at the same facility, to perform data
acquisition from experimental equipment, modeling and analysis with
supercomputers, and data analysis with either grid computing or
supercomputers.
\section{Challenges And Problems With Workflow Models and Management Systems}\label{commonFunc}

The review of different workflow models and management systems in
\S\ref{workflows} illustrates the diversity of solutions, the lack of a coherent
understanding of workflows per se, and the absence of a coordinated search for
higher level concepts in spite of very good past efforts. That is, there is no
``Standard Model'' that describes what a workflow is, the common elements of
workflow managements systems, or the description of how the pieces of such a
system interact to execute a workflow. Furthermore, there are few examples of
interoperability between existing systems in spite of significant community
pressure and calls for cross-system workflow execution. Poor or non-existent
interoperability is almost certainly a consequence of the ``Wild West'' state of
the field.

The state of the field does not mean that there is little or no common
functionality between workflow management systems in different domains. Many
sources in the literature, including several previously cited above, indicate
that the contrary is in fact true: there is significant duplication and
commonality in this space. The overlap in these technologies is rarely discussed
on its own merits, but instead it is commonly used to create large tables
comparing different systems, as in
\cite{ferreira_da_silva_characterization_nodate}. This creates a scenario where
more effort is expending discussing \textit{how} something is accomplished
versus the arguably more important question of \textit{what} must be
accomplished. 

Elaborating on \textit{what}, some primary application (workflow) needs are:
(i) lowering the burden to  develop, (ii) extend, and (iii) transport an
application (workflow) to another resource, platform or workflow system, and
(iv) a conceptual framework or basis to decide which tools are suitable or
optimal for a given workflow.

Similarly, beyond clarity on the functional and performance capabilities of a
workflow systen, the primary needs of users and developers of workflow systems
are: (i) lower the need to develop components, (ii) determine which components
to use, re-use, (iii) minimal perturbation and refactoring when extending or
generalizing the functionality or use cases supported by a workflow system,
(iv) provide constant performance across different use-case scenarios and
scales.

It is worth noting that workflow systems are rarely developed to extract
(enhance) performance. They are more about coordinating different
functionality without loss of performance. High-performance and scalability is
not often a first order concern of general workflow systems; it may however,
be a first order concern of specialized workflow systems or specific
components, e.g, Pilot-system that is responsible for scalable and efficient
task launching and management.



A healthy balance of \textit{what} versus \textit{how} is
important, but we propose that the discussion of how particular problems are
solved in workflow science has overtaken the discussion of what must be
accomplished, thus creating two severe problems: \begin{itemize} \item A
``proliferation'' of tools that largely solve the same problem in the same way,
but with separate, competing implementations primarily delineated along domain,
as opposed to technological, boundaries.  \item A general lack of
interoperability and therefore inability to address larger scientific problems
using hybrid (combined) workflows, multi-facility workflow campaigns, or
heterogenous hardware without significant reimplementation.  \end{itemize}

These two problems are closely related: Tooling proliferation might not be a
problem, given sufficient resources, in the absence of calls for
interoperability between systems, and interoperability might not be an issue if
there were not so many existing systems! However, some of the most important
aspects of these problems remain separable and can be examined as such.

Table \ref{blocks} lists six common types of functionality that are readily
observed in workflow management systems. There are certainly additional types
of functionality that are common, but for pedagogical reasons we limit the
list to the most obvious choices that can be discovered in a quick review of
the literature previously cited.

\begin{table*}[h] \begin{tabularx}{\textwidth}{|X|X|} \hline
\textbf{Functionality} & \textbf{Description} \tabularnewline\hline Data and
Metadata Management & Management of data, metadata, and general file input and
output activities whether for internal (tracking) or external (user)
consumption.  \tabularnewline\hline Workflow Execution Engine & The primary
actor that manages the execution of the activities as provided by the workflow
description. \tabularnewline\hline Resource Management and Acquisition &
Acquisition and management of resources, whether computing or instrumentation,
required for the successful execution of the workflow. \tabularnewline\hline
Task Management & Primary subsystem for managing individual activities, tasks or
``subworkflow'' using resources provided by the task management system. This
system is sometimes, but neither often nor exclusively, part of the Workflow
Execution Engine. \tabularnewline\hline Provenance Engine & System for tracking
execution history, sources and destinations of ingested and generated artifacts,
execution metadata including status, general logging, and provenance-based
inference tools. \tabularnewline\hline Application Programming Interface (API) &
A non-functional element of most workflow management systems that is critical to
successful deployment and maintenance of the full system as well as utilization
as a tool for creating and executing workflows. \tabularnewline\hline
\end{tabularx} \caption{Functionality commonly identified in workflow management
systems.} \label{blocks} \end{table*}


A primary driver for seeking interoperability between workflows systems is the
need to address larger scientific problems that can only be solved with
workflows that require multiple systems for complete execution. 

Workflow interoperability however is not a simple or single attribute. There
are at least four distinct types of interoperability that merit discussion:
\begin{itemize} \item Workflow Interoperability - sharing workflows across
different science problems.  \item Execution Delegation - delegating the
execution of a workflow to a more capable or appropriate workflow management
system \item Workflow System Interoperability - executing workflows by
different workflow management systems.  \item Interchangeable Workflow System
Components - components that can be exchanged or used concurrently across one
or more systems.  \end{itemize} Two successful examples of limited
interoperability between workflow systems are discussed in
\cite{brooks_triquetrum:_2015} and \cite{mandal_integrating_2007}. Notably
both projects leveraged flavors of the Ptolemy framework, namely Triquetrum
and Kepler, and delegated the execution of workflows.






\section{The Solution: Common Building Blocks}\label{buildings-blocks}

The two problems detailed above are side effects of the relentless march of
progress. The traditional approach for building workflow systems has been to
build as much of the required capability as possible into the system itself,
relying very little on external services or even third party code, to address
pressing issues in one or more domains. However, history has shown that
important high-level functionality slowly moves down the software stack and
into kernels, kernel services, and system libraries. Is it better at that
point to use an existing system that requires significant time and resources
to learn, or to develop yet another workflow management system that the common
tools, implementing only the gaps instead?

The answer to this question is complicated by the fact that workflows
themselves are not what they used to be. First, new workflows are often the
representation of methodological advances and may be more pervasive,  short-
lived and wide-ranging than historical workflows. Further, they are no longer
confined to ``big science'' projects since sophisticated workflows are needed
by multiple science projects, which leads to diverse “design points” and thus
making it unlikely that ``one size fits all.''  The ability to prototype, test
and experiment with workflows at scale suggests a need for interfaces and
middleware services that enable the rapid development of resources. The
challenge is to provide these capabilities along with considerations of
usability and extensibility.

Jha and Turilli discuss this trend as it relates to workflows from a cyber-
infrastructure perspective and to existing large-scale scientific workflow
efforts, \cite{jha_building_2016}. They propose that, while historically
successful, monolithic workflow systems present many problems for users,
developers, and maintainers. Instead, they propose that a new ``Lego style''
approach might work better where individual ``building blocks'' of capability
are assembled into the final workflow management system, subsystem, or
product.

A building block is a collection of functionality commonly identified across
existing workflow systems that behaves like a logically and uniformly
addressable service. Each of the types of functionality listed in Table
\ref{blocks} could be developed, presumably through one or more community
efforts, as a building block (even the API through some programming
trickery!). Other things like programming interfaces to queuing systems,
programmable pilot systems for scheduling jobs, workload balancers, and
ensemble execution tools, among others, could be provided as well to create a
rich ecosystem of reusable and interchangeable parts.

Reusable building blocks would greatly improve both interoperability and
sustainability because they would standardize, to some degree, the programming
interfaces and backends used by workflow management systems. To the extent that
projects are willing to use common building blocks, proliferation would be fully
decoupled from interoperability. Leadership Computing Facilities would not need
to support every workflow management system, just a set of common building
blocks. This is similar to how they support third-party libraries for software
development: they do not support \textit{every code} used on these machines, but
they support a set of common libraries that the codes can use. 

There is an important practical question here: Does this mean abandoning
existing workflow management systems or redeveloping existing workflows? No, and
in fact it may be quite practical to develop building blocks based on components
of the most sophisticated workflow management systems already in existence.
Furthermore, since building blocks would naturally enable interoperability, it
is quite conceivable that a workflow that only executes on one system now may
execute on many systems in the future with little or no modification. 


\section{Discussion and The Road Ahead}\label{discussion}

This is a practice (experience) paper motivated by the widely shared
perception if not strong empirical evidence/observation  that  there is a
problem in the current practice of workflow management systems.

There is an important separation between challenges of expressing workflows
effectively (algorithm) versus a workflow system that will execute the
workflow (algorithm). This paper is about the  the practice of using workflow
systems and not expressing workflows effectively. Further, it is not a
theoretically motivated or survey paper about models of workflows or workflow
systems; plenty of such papers exist which have had limited impact on the
practice of workflow systems. 

This work describes the variety of workflows, technologies, problems, and
challenges commonly found in the workflow science space.  Self-evidently, no
single workflow management system will be able to address the next generation
of scientific challenges and practical experience dictates that a change is
necessary. We propose  that common Building Blocks are a promising and
practical solution. It proposes that a Building Blocks model solve problems of
system proliferation and interoperability by redeveloping common functionality
that exists in workflow management systems into reusable services.

An important and critical test will be to devise a validation (or negation)
test for hypothesis that a building blocks approach to workflows is in fact
more scalable, sustainable and better practice than monolithic workflow
systems. We do not harbour illusions that it will not be easy, or that it is
necessarily even possible.

It is illustrative if not instructive to understand the ecosystem of the
ABDS/Cloud Model, where there are many seemingly similar components for  data-
intensive workflows. The proliferation of components suggests there is an
strong preference of functional specialization and diversity of use, as
opposed to interoperability.  Equivalentally, there is a strong binding of
components to platforms.


\section{Acknowledgements} The authors are grateful for the assistance and
support of the following people and institutions without which this work would
not have been possible.

Mr. Billings is especially grateful for the feedback provided on
\S\ref{workflows} and \S\ref{commonFunc} by his PhD committee, including Jack Dongarra, John Drake, Mike Guidry, Mallikarjun Shankar, and John Turner. Mr.
Billings would also like to acknowledge the thoughtful discussions with Jim
Belak on the nature of workflows in the ExAM project, and Robert Clay, Dan
Laney, and David Montoya on modeling and simulation workflows.

This work has been supported by the US Department of Energy, the Oak Ridge National Laboratory (ORNL) Director's Research and Development Fund, and by the ORNL
Undergraduate Research Participation Program, which is sponsored by ORNL and
administered jointly by ORNL and the Oak Ridge Institute for Science and
Education (ORISE). ORNL is managed by UT-Battelle, LLC, for the US Department
of Energy under contract no. DE-AC05-00OR22725. ORISE is managed by Oak Ridge
Associated Universities for the US Department of Energy under contract no.
DE-AC05-00OR22750.


% The bibliography 
\bibliographystyle{ACM-Reference-Format}
\bibliography{bib}

\end{document}
