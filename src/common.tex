\section{Challenges And Problems With Workflow Models and Management Systems}\label{commonFunc}

The review of different workflow models and management systems in
\S\ref{workflows} illustrates the diversity of solutions, the lack of a coherent
understanding of workflows per se, and the absence of a coordinated search for
higher level concepts in spite of very good past efforts. That is, there is no
``Standard Model'' that describes what a workflow is, the common elements of
workflow managements systems, or the description of how the pieces of such a
system interact to execute a workflow. Furthermore, there are few examples of
interoperability between existing systems in spite of significant community
pressure and calls for cross-system workflow execution. Poor or non-existent
interoperability is almost certainly a consequence of the ``Wild West'' state of
the field.

The state of the field does not mean that there is little or no common
functionality between workflow management systems in different domains. Many
sources in the literature, including several previously cited above, indicate
that the contrary is in fact true: there is significant duplication and
commonality in this space. The overlap in these technologies is rarely discussed
on its own merits, but instead it is commonly used to create large tables
comparing different systems, as in
\cite{ferreira_da_silva_characterization_nodate}. This creates a scenario where
more effort is expending discussing \textit{how} something is accomplished
versus the arguably more important question of \textit{what} must be
accomplished. A healthy balance of \textit{what} versus \textit{how} is
important, but we propose that the discussion of how particular problems are
solved in workflow science has overtaken the discussion of what must be
accomplished, thus creating two severe problems: \begin{itemize} \item A
``proliferation'' of tools that largely solve the same problem in the same way,
but with separate, competing implementations primarily delineated along domain,
as opposed to technological, boundaries.  \item A general lack of
interoperability and therefore inability to address larger scientific problems
using hybrid (combined) workflows, multi-facility workflow campaigns, or
heterogenous hardware without significant reimplementation.  \end{itemize}

These two problems are closely related: Tooling proliferation might not be a
problem, given sufficient resources, in the absence of calls for
interoperability between systems, and interoperability might not be an issue if
there were not so many existing systems! However, some of the most important
aspects of these problems remain separable and can be examined as such.

{\bf some suggestions to  build upon "what"} 

The primary application (workflow) needs are: lowering the burden to (i) develop, (ii)extend, and (iii) transport to another platform....platform The primary workflow system needs are: (i) determine when to use or re-use a workflow system, (ii) lower the need to develop components afresh....

.... deciding which tools is suitable or optimal. Monolithic components are
not conducive for functional specialization with minimal perturbation or
refactoring.

It is worth noting that workflow systems are rarely developed to extract
(enhance) performance. They are more about coordinating different
functionality without loss of performance. High-performance and scalability is
not often a first order concern of general workflow systems; it may however,
be a first order concern of specialized workflow systems or specific
components, e.g, Pilot-system that is responsible for scalable and efficient
task launching and management.

Table \ref{blocks} lists six common types of functionality that are readily
observed in workflow management systems. There are certainly additional types
of functionality that are common, but for pedagogical reasons we limit the
list to the most obvious choices that can be discovered in a quick review of
the literature previously cited.

\begin{table*}[h] \begin{tabularx}{\textwidth}{|X|X|} \hline
\textbf{Functionality} & \textbf{Description} \tabularnewline\hline Data and
Metadata Management & Management of data, metadata, and general file input and
output activities whether for internal (tracking) or external (user)
consumption.  \tabularnewline\hline Workflow Execution Engine & The primary
actor that manages the execution of the activities as provided by the workflow
description. \tabularnewline\hline Resource Management and Acquisition &
Acquisition and management of resources, whether computing or instrumentation,
required for the successful execution of the workflow. \tabularnewline\hline
Task Management & Primary subsystem for managing individual activities, tasks or
``subworkflow'' using resources provided by the task management system. This
system is sometimes, but neither often nor exclusively, part of the Workflow
Execution Engine. \tabularnewline\hline Provenance Engine & System for tracking
execution history, sources and destinations of ingested and generated artifacts,
execution metadata including status, general logging, and provenance-based
inference tools. \tabularnewline\hline Application Programming Interface (API) &
A non-functional element of most workflow management systems that is critical to
successful deployment and maintenance of the full system as well as utilization
as a tool for creating and executing workflows. \tabularnewline\hline
\end{tabularx} \caption{Functionality commonly identified in workflow management
systems.} \label{blocks} \end{table*}


A primary driver for seeking interoperability between workflows systems is the
need to address larger scientific problems that can only be solved with
workflows that require multiple systems for complete execution. 

Workflow interoperability however is not a simple or single attribute. There
are at least four distinct types of interoperability that merit discussion:
\begin{itemize} \item Workflow Interoperability - sharing workflows across
different science problems.  \item Execution Delegation - delegating the
execution of a workflow to a more capable or appropriate workflow management
system \item Workflow System Interoperability - executing workflows by
different workflow management systems.  \item Interchangeable Workflow System
Components - components that can be exchanged or used concurrently across one
or more systems.  \end{itemize} Two successful examples of limited
interoperability between workflow systems are discussed in
\cite{brooks_triquetrum:_2015} and \cite{mandal_integrating_2007}. Notably
both projects leveraged flavors of the Ptolemy framework, namely Triquetrum
and Kepler, and delegated the execution of workflows.





