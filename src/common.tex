\section{Common Functionality}\label{commonFunc}

20170413
\begin{itemize}
\item I feel like the goal here needs to be to really setup why we think that 1) we have a problem, 2) why interoperability matters, and 3) why building blocks are the solution. I like that laying it out that way gives us two problems and then one solution.
\item We need to unpack the ideal of interoperability into all of the different types.
\item experiments - can we do something around ICE?
\item Can we validate conceptual ideas by getting others involved to come up with building block types? What are the 5,6,10,...,n building blocks? Need them to tell us *why* these are building blocks for them!
\end{itemize}

Table 1 is a very simple set of rows where the first column would just be the name of the building block - carefully thought out - and the description of the block. Need to assign a function description to the component. ($n \times 2$)

Table 2 has building blocks as the rows and the columns are the WMSes. ($n \times m$)

Let $n = 6$, as the initial building block number.
Launch
AAA
Data management storage services
Reproducibility and provenance
Service/Programming/Execution/Deployment model - how are the blocks exposed to the end user

\textbf{Bring back notes on other building blocks! Look at ORNL list.}

20170418

Vertical integration = extensibility
Horizontal integration = interoperability

Building blocks - functional vs. non-functional. So provenance removed from the list below.

{\bf First cut at all lists combined}

These are neither necessary nor sufficient building blocks, but they represent the landscape of requirements:

\begin{itemize}
\item Functional
\begin{enumerate}
\item File and Meta-data Management
\item Execution Engine - Responsible for executing and controlling workflows, tasks, and subprocesses of tasks.
\item Resource Management and Acquisition - Historically includes compute, data, and network, but now includes web services and in-situ modification in menu and in-transit computing.
\item Task management - Responsible for managing the states that the tasks might be in.
\item Monitoring - Includes execution and performance monitoring of things executed by the execution engine. (Consuming building block.)
\item Logging and information - Responsible for providing services for producing information that is consumed by logs and monitors. (Producer building block.)
\end{enumerate}
\item Non-Functional
\begin{enumerate}
\item API 
\item Developer Tools (?)
\end{enumerate}
\end{itemize}

What about viz?

20170616

\begin{itemize}
    \item Reproducibility - Instrumentation and tooling. Also, any notes from Jay's comprehensive exam?
    \item Adaptive Execution/Autonomous Steering - Has not been possible to support adaptive execution with non-adaptive tools. Being more responsive to changes in workflows per se based on new information gathered at runtime, not node availability or human interaction.
\end{itemize}

\begin{table*}[h]
\begin{tabularx}{\textwidth}{|X|X|}
\hline
\textbf{Building Block Name} & \textbf{Description} \tabularnewline\hline
File and Metadata Management & desc
 \tabularnewline\hline 
Execution Engine & desc \tabularnewline\hline
Resource Management and Acquisition & desc \tabularnewline\hline 
Task Management & desc \tabularnewline\hline
Monitoring & desc \tabularnewline\hline 
Logging and Information & desc \tabularnewline\hline
Application Programming Interface (API) & desc, A non-functional block. \tabularnewline\hline
\end{tabularx}
\caption{Commonly encountered building blocks.}
\label{blocks}
\end{table*}

\begin{table*}[h]
\begin{tabularx}{\textwidth}{|X|X|X|X|X|X|}
\hline
\textbf{File and Metadata Management} & \textbf{Kepler} & \textbf{Triquetrum} & \textbf{Eclipse ICE} & \textbf{Pegasus} & \textbf{Swift}
 \tabularnewline\hline 
Execution Engine & desc &&&& \tabularnewline\hline
Resource Management and Acquisition & desc &&&& \tabularnewline\hline 
Task Management & desc &&&& \tabularnewline\hline
Monitoring & desc &&&& \tabularnewline\hline 
Logging and Information & desc &&&& \tabularnewline\hline
Application Programming Interface (API) & desc, A non-functional block. &&&& \tabularnewline\hline
\end{tabularx}
\caption{Names of ``roughly equivalent subsystems'' in several workflow engines for each proposed building block type.}
\label{blocks}
\end{table*}