\section{Challenges And Problems With Workflow Models and Management Systems}\label{commonFunc}

The review of different workflow models and management systems in
\S\ref{workflows} illustrates the diversity of solutions, the lack of a coherent
understanding of workflows per se, and the absence of a coordinated search for
higher level concepts in spite of very good past efforts. That is, there is no
``Standard Model'' that describes what a workflow is, the common elements of
workflow managements systems, or the description of how the pieces of such a
system interact to execute a workflow. Furthermore, there are few examples of
interoperability between existing systems in spite of significant community
pressure and calls for cross-system workflow execution. Poor or non-existent
interoperability is almost certainly a consequence of the ``Wild West'' state of
the field.

The state of the field does not mean that there is little or no common
functionality between workflow management systems in different domains. Many
sources in the literature, including several previously cited above, indicate
that the contrary is in fact true: there is significant duplication and
commonality in this space. The overlap in these technologies is rarely discussed
on its own merits, but instead it is commonly used to create large tables
comparing different systems, as in
\cite{ferreira_da_silva_characterization_nodate}. This creates a scenario where
more effort is expending discussing \textit{how} something is accomplished
versus the arguably more important question of \textit{what} must be
accomplished. A healthy balance of \textit{what} versus \textit{how} is
important, but we propose that the discussion of how particular problems are
solved in workflow science has overtaken the discussion of what must be
accomplished, thus creating two severe problems: \begin{itemize} \item A
``proliferation'' of tools that largely solve the same problem in the same way,
but with separate, competing implementations primarily delineated along domain,
as opposed to technological, boundaries.  \item A general lack of
interoperability and therefore inability to address larger scientific problems
using hybrid (combined) workflows, multi-facility workflow campaigns, or
heterogenous hardware without significant reimplementation.  \end{itemize}

These two problems are closely related: Tooling proliferation might not be a
problem, given sufficient resources, in the absence of calls for
interoperability between systems, and interoperability might not be an issue if
there were not so many existing systems! However, some of the most important
aspects of these problems remain separable and can be examined as such.

\subsection{Proliferation and Common Functionality} \label{commonFunc}

The problem with the increase in the number of existing workflow management
systems have been illustrated well by reports and discussions surrounding the
future of workflow management in the Leadership Computing Facilities. The
``proliferation'' of workflow management systems and the lack of a consistent
definition of a workflow are signficant barriers to the adoption of this
technology in these facilities. The High Performance Computing Facility
Operational Assessment 2015: Oak Ridge Leadership Computing Facility (OLCF)
report, \cite{barker_scientific_2007}, describes the problem that such
facilities face, \begin{displayquote} These discussions concluded with the
observation that the current proliferation of workflow systems in response to
perceived domain-specific needs of scientific workflows makes it difficult to
choose a site-wide operational workflow manager, particularly for the
leadership-class machines. However, there are opportunities where facilities can
centralize workflow technology offerings to reduce anticipated fragmentation.
This is especially true if a facility attempts to develop, deploy, and operate
each and every workflow solution requested by the user community. Through these
evaluations, the OLCF seeks to identify interesting intersections that are of
the most value to OLCF stakeholders.  \end{displayquote} The strategy of the
OLCF is notable because it makes a very practical observation that the problem
of proliferation can be solved by consolidation of common functionality. This is
typical of an operational perspective where deployment of capability is more
important than in-depth investigation and research into how that capability
functions. 

Table \ref{blocks} lists six common types of functionality that are readily
observed in workflow management systems. There are certainly additional types of
functionality that are common, but for pedagogical reasons we limit the list to
the most obvious choices that can be discovered in a quick review of the
literature previously cited.

\begin{table*}[h] \begin{tabularx}{\textwidth}{|X|X|} \hline
\textbf{Functionality} & \textbf{Description} \tabularnewline\hline Data and
Metadata Management & Management of data, metadata, and general file input and
output activities whether for internal (tracking) or external (user)
consumption.  \tabularnewline\hline Workflow Execution Engine & The primary
actor that manages the execution of the activities as provided by the workflow
description. \tabularnewline\hline Resource Management and Acquisition &
Acquisition and management of resources, whether computing or instrumentation,
required for the successful execution of the workflow. \tabularnewline\hline
Task Management & Primary subsystem for managing individual activities, tasks or
``subworkflow'' using resources provided by the task management system. This
system is sometimes, but neither often nor exclusively, part of the Workflow
Execution Engine. \tabularnewline\hline Provenance Engine & System for tracking
execution history, sources and destinations of ingested and generated artifacts,
execution metadata including status, general logging, and provenance-based
inference tools. \tabularnewline\hline Application Programming Interface (API) &
A non-functional element of most workflow management systems that is critical to
successful deployment and maintenance of the full system as well as utilization
as a tool for creating and executing workflows. \tabularnewline\hline
\end{tabularx} \caption{Functionality commonly identified in workflow management
systems.} \label{blocks} \end{table*}

\subsection{Interoperability}

The primary driver for seeking interoperability between workflows systems is the
need to address larger scientific problems that can only be solved with
workflows that require multiple systems for complete execution. There have been
a number of community calls for interoperability. For example, Session IV of the
Twentieth Anniversary Meeting of the SOS Workshop (SOS20) focused on workflow
and workflow management system development activities of the three participating
institutions: Sandia National Laboratory, Oak Ridge National Laboratory, and the
Swiss National Supercomputing Centre, \cite{pack_sos20_2016}. Multiple
presenters illustrated the challenges facing the workflow science community and
widely agreed that no single workflow management system could satisfy all the
needs of those present. Instead, attendees proposed that the community as a
whole would be served best by seeking to enable interoperability where possible. 

The Leadership Computing Facilities, as in \S\ref{commonFunc}, are affected by
this as well. Consider the possibility that every facility may end up supporting
different workflows systems entirely such that workflows at one facility can not
be run at another without significant work to install one or more additional
workflow management systems! This idea is also illustrated well in The Future of
Scientific Workflows report through the concept of the ``large-scale science
campaign,'' \cite{deelman_future_2015}. Such a campaign integrates multiple
workflows, not necessarily all in the same workflow management system or at the
same facility, to perform data acquisition from experimental equipment, modeling
and analysis with supercomputers, and data analysis with either grid computing
or supercomputers.

There are at least four distinct types of interoperability that merit
discussion: \begin{itemize} \item Workflow Interoperability - sharing workflows
across different science problems.  \item Execution Delegation - delegating the
execution of a workflow to a more capable or appropriate workflow management
system \item Workflow System Interoperability - executing workflows by different
workflow management systems.  \item Interchangeable Workflow System Components -
components that can be exchanged or used concurrently across one or more
systems.  \end{itemize} Two successful examples of limited interoperability
between workflow systems are discussed in \cite{brooks_triquetrum:_2015} and
\cite{mandal_integrating_2007}. Notably both projects leveraged flavors of the
Ptolemy framework, namely Triquetrum and Kepler, and delegated the execution of
workflows.
