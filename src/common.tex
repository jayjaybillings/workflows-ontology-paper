\section{Challenges And Problems With Workflow Models and Management Systems}\label{commonFunc}

The review of different workflow models and management systems in
\S\ref{workflows} illustrates the diversity of solutions, the lack of a coherent
understanding of workflows per se, and the absence of a coordinated search for
higher level concepts in spite of very good past efforts. That is, there is no
``Standard Model'' that describes what a workflow is, the common elements of
workflow managements systems, or the description of how the pieces of such a
system interact to execute a workflow. Furthermore, there are few examples of
interoperability between existing systems in spite of significant community
pressure and calls for cross-system workflow execution. Poor or non-existent
interoperability is almost certainly a consequence of the ``Wild West'' state of
the field.

The state of the field does not mean that there is little or no common
functionality between workflow management systems in different domains. Many
sources in the literature, including several previously cited above, indicate
that the contrary is in fact true: there is significant duplication and
commonality in this space. The overlap in these technologies is rarely discussed
on its own merits, but instead it is commonly used to create large tables
comparing different systems, as in
\cite{ferreira_da_silva_characterization_nodate}. This creates a scenario where
more effort is expending discussing \textit{how} something is accomplished
versus the arguably more important question of \textit{what} must be
accomplished. 

Elaborating on \textit{what}, some primary application (workflow) needs are:
(i) lowering the burden to  develop, (ii) extend, and (iii) transport an
application (workflow) to another resource, platform or workflow system, and
(iv) a conceptual framework or basis to decide which tools are suitable or
optimal for a given workflow.

Similarly, beyond clarity on the functional and performance capabilities of a
workflow systen, the primary needs of users and developers of workflow systems
are: (i) lower the need to develop components, (ii) determine which components
to use, re-use, (iii) minimal perturbation and refactoring when extending or
generalizing the functionality or use cases supported by a workflow system,
(iv) provide constant performance across different use-case scenarios and
scales.

It is worth noting that workflow systems are rarely developed to extract
(enhance) performance. They are more about coordinating different
functionality without loss of performance. High-performance and scalability is
not often a first order concern of general workflow systems; it may however,
be a first order concern of specialized workflow systems or specific
components, e.g, Pilot-system that is responsible for scalable and efficient
task launching and management.

A healthy balance of \textit{what} versus \textit{how} is
important, but we propose that the discussion of how particular problems are
solved in workflow science has overtaken the discussion of what must be
accomplished, thus creating two severe problems: \begin{itemize} \item A
``proliferation'' of tools that largely solve the same problem in the same way,
but with separate, competing implementations primarily delineated along domain,
as opposed to technological, boundaries.  \item A general lack of
interoperability and therefore inability to address larger scientific problems
using hybrid (combined) workflows, multi-facility workflow campaigns, or
heterogenous hardware without significant reimplementation.  \end{itemize}

These two problems are closely related: Tooling proliferation might not be a
problem, given sufficient resources, in the absence of calls for
interoperability between systems, and interoperability might not be an issue if
there were not so many existing systems! However, some of the most important
aspects of these problems remain separable and can be examined as such.

Workflow interoperability however is not a simple or single attribute. There
are at least four distinct types of interoperability that merit discussion:

\begin{itemize}

\item Workflow Interoperability - sharing workflows across different science
problems. This was an original motivation in the initial days of eScience  and
reproducible computational science. Early projects such as the MyGrid
(subsequently MyExperiment) and related projects, pioneered and advanced the
ability to share workflows across science domains, science  problems and
scientists.

\item Execution Delegation - delegating the execution of a workflow to a more
capable or appropriate workflow management system. The formal specification of
a workflow as a directed acyclic graph (DAG) and associated data descriptions, 
such that the specification was complete and thereby in principle executable
by any capable workflow management system.  Although in principle and
conceptually easy, this has proven to be less successful in practice. At least
two primary reasons: (i) DAG are common, but not universal formal specifications
of workflows, and (ii) many specific consideration and assumptions beyond those
associated with a DAG need to be factored when executing workflows. These
assumptions and specific considerations in turn are often due to inadequate
infrastructure abstraction and separation of concerns. 

\item Workflow System Interoperability - executing the same workflow(s) by
different workflow management systems. In addition to the absence of a
technical for formal basis for designing workflow management systems, the
sociology of software engineering and tooling contributed to the proliferation
of workflow management systems. In the presence of proliferation of tools,
there was always a principled if not a practical demand for such workflow
system interoperability. However, even if initially a  more  "principled" form
of interoperability, it can be argued that Workflow System Interoperabilty is
increasingly important due to the needs and requirements of reproducible
science.

\item Interchangeable Workflow System Components - components that can be
exchanged or used concurrently across one or more systems. hitherto, this is
the least articulated or argued form of interoperability. However, it is
the most critical and core form of interoperability that our work suggests 
must be addressed, if the component based approach to workflow systems is
ever to supplement monolithic workflow systems.

\end{itemize} 


Two successful examples of limited
interoperability between workflow systems are discussed in
\cite{brooks_triquetrum:_2015} and \cite{mandal_integrating_2007}. Notably
both projects leveraged flavors of the Ptolemy framework, namely Triquetrum
and Kepler, and delegated the execution of workflows. A primary driver for
seeking interoperability across these workflows systems has been the need to
address larger scientific problems that can only be solved with workflows that
require multiple systems for complete execution.






