\section{The Solution: Common Building Blocks}\label{buildings-blocks}

For many reasons, the traditional approach for building workflow systems
has been to build as much of the required capability as possible into
the system itself, relying very little on external services or even
third party code. This does not track will with the course of history
since high-level functionality tends to slowly creep down the
software stack and into kernels, kernel services, and system libraries. Jha and
Turilli discuss this trend as it relates to workflows from a cyber
infrastructure perspective and to existing large scale scientific
workflow efforts, \cite{jha_building_2016}. They propose that, while historically
successful, monolithic workflow systems present many problems for users,
developers, and maintainers. Instead, they propose that a new ``Lego
style'' approach might work better where individual ``building blocks''
of capability are assembled into the final workflow product. These
building blocks would include things like programming interfaces to
queuing systems, programmable pilot systems for scheduling jobs,
workload balancers, and ensemble execution tools, among others.

This approach would greatly improve both interoperability and sustainability
because it would standardize the programming interfaces and backends used by
workflow management systems. Jha et al. are developing a white paper to address
this further, \cite{jha_towards_2016}.