\section{The Diversity of Workflow Models}\label{workflows}

One of the most challenging aspects of studying workflows is the way the
vocabulary has been overloaded unintentionally.  It is somewhat clearer to
understand by starting from a historial perspective.

The use and study of workflows and the initial implementation of workflow
management systems, i.e., systems that manage one or more activities related to
workflows, and especially workflow execution, developed in the business world
with the need to automate business processes. Lud\"{a}scher et al.  ascribe the
origins of workflows and workflow management systems to ``office automation''
trends in the 1970s, \cite{ludascher_scientific_2006}. Van Der Aalst argues that
``workflows'' arose from the needs of businesses to not only execute tasks, but
``to manage the flow of work through the organization,'' and that managing
workflows is the natural evolution from the monolithic applications of the 1960s
to applications that rely on external functionality in the 1990s,
\cite{van_der_aalst_application_1998}. By 1995, in the presence of many workflow
tools, the Workflow Management Coalition had developed a ``standard'' definition
of workflows, \cite{hollingsworth_workflow_1993},

\begin{displayquote} A Workflow is the automation of a business process, in
whole or part, during which documents, information or tasks are passed from one
participant (a resource; human or machine) to another for action, according a
set of procedural rules.  \end{displayquote}

In the early 2000s, workflow systems started finding use in scientific contexts
where process automation was required for scientific uses instead of traditional
business uses. The focus of scientific workflows, at the time, also shifted to
focus primarily on data processing and managing heterogenous infrastructure for
large ``grids'' of networked services,
\cite{yu_taxonomy_2005}. Yu and Buyya define a workflow as

\begin{displayquote} ... a collection of tasks that are processed on distributed
resources in a well-defined order to accomplish a specific goal.

\end{displayquote}

This latter definition is important because of what is missing: the human
element. For many in the grid/eScience workflows community this has become the
standard definition of a workflow and the involvement of humans results not in a
single workflow, but multiple workflows spanned by a human.  Machines or
instruments are absent from the definition as well, but in practice many modern
grid workflows are launched automatically when data ``comes off'' of instruments
because they remain the primary source of data in grid workflows, (c.f. -
\cite{megino_panda:_2015}).

In addition to ``grid workflows,'' the scientific community started exploring
``modeling and simulation workflows'' which focus not on data flow, but on the
orchestration of activities related to modeling and simulation instead,
sometimes on small local computers, but often on the largest of the world's
``Leadership Class'' supercomputers. Unlike grid workflows they tend to require
human interaction in one way or another.  Some of these workflows are defined in
the context of a particular way of working, such as the Automation, Data,
Environment, and Sharing (ADES) model of Pizzi et al., \cite{pizzi_aiida:_2016},
the ``Design-to-Analysis'' model of Clay et al., \cite{clay_incorporating_2015},
or the model of Billings et al.,\cite{billings_eclipse_2017}.

Additional types of workflows in the scientific community include workflows that
process ensembles of calculations for uncertainty quantification, verification
and validation or probabilistic risk assessment, \cite{montoya_apex_2016}, and
workflows that are used for testing software. These workflows share the property
that they are all running a very large set of coordinated jobs that only provide
value when run together. However, they differ because testing workflows
typically run each test as an independent task, whereas the other workflows may
or may not change the tasks that are executed based on the intermediate state of
the entire ensemble. These workflows require a large cluster or possibly a
supercomputer in extreme cases.

Many scientific workflows have been hard-coded into dedicated environments - not
general purpose workflow management systems - that serve as point solutions
developed for the sole purpose of that single well-defined workflow, or at most
a few, to meet the needs of a single community. This leads to an important
defining characteristic for workflow management systems versus the point
solutions: workflow management systems are extensible through a public
Application Programming Interface or other method and extension does not, in
general, require the intervention of the original author. ``Embedding'' workflows
into point-solutions may be the best solution in many cases, but the
distinction between point-solutions and full workflow management systems is
important because it clearly demonstrates  that some parties prefer to focus on
rapidly creating new or modifying old workflows while others may only be
interested in executing well-defined, very stable workflows.

Finally, an important class of scientific workflows is the set of ``conceptual
workflows'' that broadly define activities based on the policies of a given
community. These are common in large collaborations such as the Community Earth
System Model (CESM), \cite{noauthor_cesm_nodate}. These workflows describe a
series of activities that contain both human- and computer-controlled tasks
and look like business workflows. However, the level of detail tends to
oscillate between very high and very low, as does the degree of abstraction,
depending on the author. These workflows are important because they are often
referred to in the same discussions as the other types of workflows described
above. This illustrates the important fact that not all scientific workflows are
machine executable, and it may be impossible to automate them in a workflow
management system, even one that is very good at defining abstract workflows. It
also demonstrates the difficulties that can arise in a discussion on workflows
because of ambiguity in the definition.

\subsection{Taxonomies and Classification}\label{taxonomies-and-classification}
There have been several efforts to classify, survey or develop taxonomies for
workflows and workflow management systems and these efforts are significant in
large part because they represent a collective call for higher order concepts in
the space. Yu and Buyya present an exceptional and noteworthy taxonomy for grid
workflows. Multiple other efforts provide highly useful vocabularies and
analyses as well.

Yu and Buyya developed a taxonomy for workflow management systems on grids that
sought to capture the architectural style while identifying comparison criteria,
\cite{yu_taxonomy_2005}. Their work is notable because it largely avoids a
discussion of applications and focuses purely on the functional properties of
the workflow management systems as they exist on the grids. Their work also
shows how thirteen common grid workflow management systems, including Pegasus
and Kepler, are covered by the taxonomy. Like other authors, Yu and Buyya cite
the lack of standardized workflow syntax and language as sources of
interoperability issues.

Scientific workflow management systems have flourished since their inception,
although not without significant overlap and duplication of effort. The survey
of scientific workflow management systems by Barker and Hemert illustrates both
growth and growing pains while also providing important observations and
recommendations on the topic, \cite{barker_scientific_2007}.

Barker and Hemert also provide key insights into the history of workflow
management systems as an important part of business automation. The authors make
an important comparison between traditional business workflow management systems
and their scientific counterparts, citing in particular that traditional
business workflow tools employ the wrong abstraction for scientists They define
workflows using the ``standard'' definition from the Workflow Management
Coalition, (c.f. \S \ref{workflows} above).

The discussion points that Barker and Hemert raise are important because of
their continuing importance and relevance today, particularly the need to enable
programmability through standard languages instead of custom, proprietary
languages. Sticking to standards is important and perhaps illustrated best
by Barker's and Hemert's statement that

\begin{displayquote} If software development and tool support terminates on one
proprietary framework, workflows will need to be re-implemented from scratch.
\end{displayquote}

This is an important point even for workflow tools that do not use proprietary
standards, but ``roll their own'' solutions. What can be done to support those
tools and reproduce those workflows once support for continued development ends?

Montoya et al. discuss workflow needs for the Alliance for Application
Performance at Extreme Scale (APEX), \cite{nersc_apex_2016}, and describe three
main classes of workflows: Simulation Science, Uncertainty Quantification (UQ),
and High Throughput Computing (HTC), \cite{montoya_apex_2016}.  HTC workflows
start with the collection of data from experiments that is in turn transported
to large compute facilities for processing. Many grid workflows are HTC
workflows, but not all HTC workflows are grid workflows since some HTC workflows
- such as those presented by Montoya et al. - may be run on large
resources that are not traditionally ``grid machines.'' When Montoya et al.
describe scientific workflows, they are refering to the modeling and simulation
workflows described above. Montoya et al. also provide a detailed mapping of
each workflow type to optimal hardware resources for the APEX program.

The U.S. Department of Energy sponsored the \emph{DOE NGNS/CS Scientific
Workflows Workshop} on April 20-21st 2015. In the report, Deelman et al.
describe the requirements and research directions for scientific workflows for
the exascale environment, \cite{deelman_future_2015}\cite{deelman_future_2017}.
The report (and paper) describes scientific workflows primarily by three
application types: Simulations, Instruments, and Collaborations. The findings of
the workshop are comprehensive and encouraging, with recommendations for
research priorities in Application Requirements, Hardware Systems, System
Software, Workflow Management System Design and Execution, Programming and
Usability, Provenance Capture, Validation, and Workflow Science.

The definitions of a ``workflow'' and ``workflow management systems'' are
thoroughly explored and put into context for the purposes of the workshop. The
authors of the report are very careful to define workflows not just as a
collection of managed processes, which is common, but in such a way that it is
clear that reproducibility, mobility, and some degree of generality are required
by both the description of the workflow and the management system. (N.B. - The
report appears to provide three separate definitions for ``workflow'' on pages
6, 9 and 10.)

Ferreira da Silva et al. attempt to characterize workflow management systems in
\cite{ferreira_da_silva_characterization_nodate}. The authors reduce key
properties of workflow systems into four incongruent areas: (i) design, (ii)
execution and monitoring, (iii) reusability and (iv) collaboration. These
properties are essential considerations for most  software with limited
specificity for workflow management systems. Furthermore, there is general
conflation between classification and taxonomy and significant incoherence
between entries in equivalence classes. Most significantly, it fluctuates
somewhat chaotically between discussing workflows and workflow management
systems without linking workflow properties to successful design and properties
of workflow systems.
